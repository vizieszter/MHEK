\newcommand\pontvastagsag{1 mm} %nevezetes pontok vastagsága

\section*{Túlhevítés nélküli Rankine-Clausius-körfolyamat}
%körfolyamat rajza
\begin{figure}[h]
	\centering
	\begin{tikzpicture}
		\newcommand\COR{1.2}; %korrekciós érték az ábra késsöbbi méreteinek változtatásához
		\draw[thick] (0, 0) -- (4*\COR, 0); %legfelső vonal
		\draw[->, thick] (4*\COR, 0) -- (4*\COR, -1.5*\COR);
		\draw[thick] (4*\COR, 0*\COR) -- (4*\COR, -2*\COR); %turbinaba tartó vonal
		%turbina
		\draw[thick] (4*\COR, -2*\COR) -- (5*\COR, -2.5*\COR); %turbina
		\draw[thick] (5*\COR, -2.5*\COR) -- (5*\COR, -1*\COR);
		\draw[thick] (5*\COR, -1*\COR) -- (4*\COR, -1.5*\COR); %turbina vége
		\draw[thick] (5*\COR, -2.5*\COR) -- (5*\COR, -4*\COR); %kondenzátorba tartó vonal
		%turbina T betűje
		\node[thick] at (4.5*\COR, -1.75*\COR) {T};
		
		%kondenzátor
		\draw[thick] (5*\COR, -4.5*\COR) circle (0.5cm*\COR); %kondenzátor kör
		\draw[thick] (5.25*\COR, -4.5*\COR) -- (5*\COR,-4.35*\COR); %kondenzátor cikk-cakk
		\draw[->, thick] (5*\COR, -4.35*\COR) -- (5.9*\COR,-4.1*\COR);
		\draw[thick] (5.25*\COR, -4.5*\COR) -- (5*\COR, -4.65*\COR);
		\draw[thick] (5*\COR, -4.65*\COR) -- (5.8*\COR, -4.9*\COR); 
		%kondenzátor vége
		\draw[thick] (5*\COR, -5*\COR) -- (5*\COR, -6*\COR);
		\draw[thick] (5*\COR, -6*\COR) -- (0*\COR, -6*\COR);
		\draw[thick] (0*\COR, -6*\COR) -- (0*\COR, -4*\COR);
		%kazán
		\draw[thick] (-0.5*\COR, -4*\COR) rectangle (0.5*\COR, -2*\COR);
		\draw[thick] (0*\COR, -2*\COR) -- (0*\COR, 0*\COR);
		%kazán K betűje
		\node at (0*\COR, -3*\COR) {K};
		%gép
		\draw[thick] (5*\COR, -1.75*\COR) -- (6.5*\COR, -1.75*\COR);
		\draw[thick] (7*\COR, -1.75*\COR) circle (0.5cm*\COR);
		\draw[thick] (7*\COR, -1.25*\COR) -- (7*\COR, 0*\COR);
		\draw[thick] (6.6*\COR, -1*\COR) -- (7.4*\COR, -0.75*\COR);
		\draw[thick] (6.6*\COR, -0.75*\COR) -- (7.4*\COR, -0.5*\COR);
		\draw[thick] (6.6*\COR, -0.5*\COR) -- (7.4*\COR, -0.25*\COR);
		%gép betűje
		\node at (7*\COR, -1.75*\COR) {G};
		%pontok megrajzolása
		%1-es pont
		\draw[thick, fill = black] (0*\COR, -4.5*\COR) circle (\pontvastagsag*\COR);
		\node[circle, draw, inner sep = 1mm, thick] at (-0.5*\COR, -4.5*\COR) {1};
		%2-es pont
		\node[circle, draw, inner sep = 1mm, thick] at (1*\COR, -3*\COR) {2};
		%3-as pont
		\draw[thick, fill = black] (0*\COR, -1.5*\COR) circle (\pontvastagsag*\COR);
		\node[circle, draw, inner sep = 1mm, thick] at (-0.5*\COR, -1.5*\COR) {3};
		%4-es pont
		\draw[thick, fill = black] (5*\COR, -3.2*\COR) circle (\pontvastagsag*\COR);
		\node[circle, draw, inner sep = 1mm, thick] at (5.5*\COR, -3.2*\COR) {4};
		%nyomaték megrajzolása
		\draw [->, thick] (5.5*\COR, -2.1*\COR) arc [radius = 0.45*\COR, start angle = -130, end angle = 130];
		%q_EL bejelölése
		\draw[->] (5.7*\COR, -4.5*\COR) -- (7*\COR, -4.5*\COR) node [midway, below] {$q_{EL}$};
		%q_BE bejelölése
		\draw[->] (-2*\COR, -3*\COR) -- (-0.7*\COR, -3*\COR) node [midway, above] {$q_{BE}$};
		%munka jelölése
		\node at (5.9*\COR, -1*\COR) {$w$};
		 
	\end{tikzpicture}
	\caption{A túlhevítő nélküli körfolyamat ábrája}
\end{figure}

%diagram rajza
\begin{figure}[h]
	\centering
	\begin{tikzpicture}
		\newcommand\kicsinyites{0.7};
		% Tengelyek
		\draw[->] ({0}, {0}) -- ({14*\kicsinyites}, {0}) node[anchor=base east, shift={(0,-0.5)}]{$s$};
		\draw[->] ({0}, {0}) -- ({0}, {9*\kicsinyites}) node[anchor=north east]{$T$};
		
		\begin{axis}[
			axis lines = middle,
			axis line style = {draw = none},
			xlabel={},
			ylabel={},
			ytick=\empty,
			xtick=\empty,
			width=15cm*\kicsinyites, height=10cm*\kicsinyites, xmin=0, ymin=0
			]
			
			\addplot [thick] table {./I8FMUH/ts.txt};
			
			%2-3 vonal nyillal
			\addplot[->, thick] coordinates {(2.947, 266.85) (4.7, 266.85)};
			\addplot[thick] coordinates {(2.947, 266.85) (5.95, 266.85)};
			
			%3-4 vonal nyillal
			\addplot[thick] coordinates {(5.95, 266.85) (5.95, 56.85)};
			\addplot[->, thick] coordinates {(5.95, 266.85) (5.95, 140)};
			
			%4-1 vonal nyillal
			\addplot[thick] coordinates {(5.95, 56.85) (0.79, 56.85)};
			\addplot[->, thick] coordinates {(5.95, 56.85) (3.2, 56.85)};
			
			%1-2 vonal nyillal
			\addplot[->, thick] coordinates {(2.059, 171.850) (2.108, 176.850)};
			\addplot [thick] table {./I8FMUH/egykettogorbe.txt};
			
			%pontozás
			\draw[fill] (axis cs:0.790, 56.85) circle [radius = \pontvastagsag];
			\draw[fill] (axis cs:2.947, 266.85) circle [radius = \pontvastagsag];
			\draw[fill] (axis cs:5.95, 266.85) circle [radius = \pontvastagsag];
			\draw[fill] (axis cs:5.95, 56.85) circle [radius = \pontvastagsag];
			
			%számozás
			\node[thick, circle, draw, inner sep = 1mm] at (axis cs:0.5, 90) {1};
			\node[thick, circle, draw, inner sep = 1mm] at (axis cs:2.6, 300) {2};
			\node[thick, circle, draw, inner sep = 1mm] at (axis cs:6.5, 300) {3};
			\node[thick, circle, draw, inner sep = 1mm] at (axis cs:6.5, 50) {4};

			
		\end{axis}
	
	\end{tikzpicture}
	\caption{A túlhevítő nélküli körfolyamat diagramja}
%egyenletek beillesztése
\end{figure}
\paragraph{Bevezetett hő számítása}
	\begin{equation}
		q_{BE} = h_3 - h_1
	\end{equation}
\paragraph{Kondenzátorban elvont hő}
	\begin{equation}
		q_{EL} = - q_{4,1} = h_4 - h_1
	\end{equation}
\paragraph{Munka számítása}
	\begin{equation}
		w_t = w_{t_{3,4}} = h_3-h_4
	\end{equation}
\paragraph{Termikus hatásfok}
	\begin{equation}
		\eta_T = \frac{w}{q_{BE}} = \frac{w_t}{q_{BE}}
	\end{equation}