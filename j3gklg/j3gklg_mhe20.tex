\begin{tabular}{ | p{2cm} | p{14cm} | } 
	\hline
	Név & Albert Botond Miksa \\ 
	\hline
	Szak & Gépész mérnök alapszak  \\ 
	\hline
	Félév & 2019/2020 II. (tavaszi) félév \\ 
	\hline
\end{tabular}

\section*{A túlhevítést alkalmazó Rankine--Clausius-körfolyamat}
\addcontentsline{toc}{section}{A túlhevítést alkalmazó Rankine--Clausius-körfolyamat}

Rajzolja le a túlhevítést alkalmazó Rankine–Clausius-körfolyamat kapcsolási vázlatát, a körfolyamatot $T-s$ diagramban, elhanyagolva a tápszivattyú hatását! Jelölje be a munkát ($w$) és a kondenzátorban elvont hőt ($q_K$)! Ha mindegyik nevezetes pontban ismertek az állapotjelzők, akkor hogyan számítható a bevitt hő ($q_{BE}$), a munka ($w$), a kondenzátorban elvont hő ($q_K$) és a termikus hatásfok ($\eta_T$)?
\begin{figure}[h]
	\centering
	\label{figure:j3gklg-vgtsd}
	\begin{tikzpicture}
	% Rács és vágómaszk
	%\draw[step=1cm, gray, very thin] (-1.5, -1) grid (14.5, 11);
	%\clip (-1.5, -1) rectangle (14.5, 11);
	
	% A tengelykeresztet az axis környezet hozza létre
	\begin{axis}[
	width=16cm, height=12cm,
	xmin=0, xmax=10.8,
	ymin=0, ymax=510, 
	axis lines = middle,
	axis line style={->},
	xlabel=$s \left(\si{\kilo\joule\per\kilogram\kelvin}\right)$, 
	xlabel style={
		at=(current axis.right of origin), 
		anchor=north east
	}, 
	ylabel=$T \left(\si{\degreeCelsius}\right)$, 
	ylabel style={
		at=(current axis.above origin), 
		anchor=north east
	},
	xtick={1, 2,3,  4,5,6,7,  8 },
	ytick={100, 200, 300, 400}
	]
	
% T-s diagram

	\addplot[thick] table {./j3gklg/ts.txt};
	\addplot[ultra thick] table {./j3gklg/1-2.txt};
	\draw[ultra thick](axis cs:2.618, 230.850)-- (axis cs:6.207, 230.850);
	\addplot[ultra thick] table {./j3gklg/3-4.txt};
	\draw [ultra thick](axis cs:6.91718,	467.909)--(axis cs:6.917189, 50.850);
	\draw [ultra thick](axis cs:6.917189, 50.850)--(axis cs:0.715, 50.850);
	
%Számozások	
\node[circle, draw, inner sep = 1mm, thick] at (axis cs: 0.515, 80.850) {1};

\filldraw[blue, fill=blue] (axis cs: 0.715, 50.850) circle (1mm);

\node[circle, draw, inner sep = 1mm, thick] at (axis cs: 2.4, 250) {2};
\filldraw[blue, fill=blue] (axis cs: 2.618, 230.850) circle (1mm);

\node[circle, draw, inner sep = 1mm, thick] at (axis cs: 5.7, 260) {3};
\filldraw[blue, fill=blue] (axis cs: 6.207, 230.850) circle (1mm);

\node[circle, draw, inner sep = 1mm, thick] at (axis cs: 7.4, 467) {4};
\filldraw[blue, fill=blue] (axis cs: 6.91718,	467.909) circle (1mm);

\node[circle, draw, inner sep = 1mm, thick] at (axis cs: 7.25, 80) {5};
\filldraw[blue, fill=blue] (axis cs: 6.917189, 50.850) circle (1mm);
%Vonalkázás felső
\addplot [name path = 1-2] table {./j3gklg/1-2.txt};
\addplot [name path = 3-4] table {./j3gklg/3-4.txt};
\addplot [name path = egyenes1] coordinates {(6.91718, 50.85) (6.91718, 230.850)};
\addplot [name path = egyenes2] coordinates {(6.91718, 230.850) (6.91718, 467.909)};
\addplot[pattern=north west lines ] fill between [  of=1-2 and egyenes1,];
\addplot[pattern=north west lines] fill between [  of=3-4 and egyenes2];


% Vonalkázás alsó
\fill[pattern={north east lines}] (axis cs:0.715, 50.850) rectangle (axis cs:6.917189, 0);
%Jelölés

\draw[o-] (axis cs:6, 150) -- (axis cs:9, 250) node [ above] {$w$};
\draw[o-] (axis cs:6.5, 30) -- (axis cs:9, 100) node [ above] {$q_{K}$};

%Fekete vonalak eltűntetése 
\addplot[thick] table {./j3gklg/ts.txt};
\addplot[blue,ultra thick,mid arrow=blue] table {./j3gklg/1-2.txt};
\draw[blue,ultra thick,mid arrow=blue](axis cs:2.618, 230.850)-- (axis cs:6.207, 230.850);
\addplot[blue,ultra thick,mid arrow=blue] table {./j3gklg/3-4.txt};
\draw [blue,ultra thick,mid arrow=blue](axis cs:6.91718,	467.909)--(axis cs:6.917189, 50.850);
\draw [blue,ultra thick,mid arrow=blue](axis cs:6.917189, 50.850)--(axis cs:0.715, 50.850);
	\end{axis}

	
 
	
	
	\end{tikzpicture}
	\caption{Rankine--Clausius-körfolymat ábrája víz-gőz $T-s$ diagramban}
\end{figure}

%Kapcsolási vázlat

\begin{figure}[h]
	\pgfmathsetmacro{\pontvastagsag}{0.1} %nevezetes pontok vastagsága
	\centering
	\begin{tikzpicture}
	\newcommand\COR{1.2}; %korrekciós érték az ábra késsöbbi méreteinek változtatásához
	\draw[thick,mid arrow=black] (0, 0) -- (4*\COR, 0); %legfelső vonal
	\draw[thick] (4*\COR, 0) -- (4*\COR, -1.5*\COR);
	\draw[thick] (4*\COR, 0*\COR) -- (4*\COR, -2*\COR); %turbinaba tartó vonal
	%turbina
	\draw[thick] (4*\COR, -2*\COR) -- (5*\COR, -2.5*\COR); %turbina
	\draw[thick] (5*\COR, -2.5*\COR) -- (5*\COR, -1*\COR);
	\draw[thick] (5*\COR, -1*\COR) -- (4*\COR, -1.5*\COR); %turbina vége
	\draw[thick] (5*\COR, -2.5*\COR) -- (5*\COR, -4*\COR); %kondenzátorba tartó vonal
	%turbina T betűje
	\node[thick] at (3.5*\COR, -1.75*\COR) {T};
	
	%kondenzátor
	\draw[thick] (5*\COR, -4.5*\COR) circle (0.5cm*\COR); %kondenzátor kör
	\draw[thick] (5.25*\COR, -4.5*\COR) -- (5*\COR,-4.35*\COR); %kondenzátor cikk-cakk
	\draw[->, thick] (5*\COR, -4.35*\COR) -- (5.9*\COR,-4.1*\COR);
	\draw[thick] (5.25*\COR, -4.5*\COR) -- (5*\COR, -4.65*\COR);
	\draw[thick] (5*\COR, -4.65*\COR) -- (5.8*\COR, -4.9*\COR); 
	%kondenzátor vége
	\draw[thick] (5*\COR, -5*\COR) -- (5*\COR, -6*\COR);
	\draw[thick,mid arrow=black] (5*\COR, -6*\COR) -- (0*\COR, -6*\COR);
	\draw[thick] (0*\COR, -6*\COR) -- (0*\COR, -5*\COR);
	%kondenzátor S betűje
	\node[thick] at (6.3*\COR, -4*\COR) {S};
	%kazán
	\draw[thick] (-0.5*\COR, -5*\COR) rectangle (0.5*\COR, -3*\COR);
	\draw[thick] (0*\COR, -3*\COR) -- (0*\COR, -1.8*\COR);
	\draw[thick](0*\COR, -2.3*\COR) -- (0*\COR,-1.8*\COR);
	%kazán K betűje
	\node at (1*\COR, -4*\COR) {K};
	%túlhevítő 
	\draw[thick] (0*\COR,-1.8*\COR) -- (-0.2*\COR, -1.7*\COR);
	\draw[thick] (-0.2*\COR, -1.7*\COR) -- (0.2*\COR, -1.5*\COR);
	\draw[thick] (0.2*\COR, -1.5*\COR) -- (-0.2*\COR, -1.3*\COR);
	\draw[thick] (-0.2*\COR, -1.3*\COR) -- (0.2*\COR, -1.1*\COR);
	\draw[thick] (0.2*\COR, -1.1*\COR) -- (-0.2*\COR, -0.9*\COR);
	\draw[thick] (-0.2*\COR, -0.9*\COR) -- (0.2*\COR, -0.7*\COR);
	\draw[thick] (0.2*\COR, -0.7*\COR) -- (-0.2*\COR, -0.5*\COR);
	\draw[thick] (-0.2*\COR, -0.5*\COR) -- (0.2*\COR, -0.3*\COR);
	\draw[thick] (0.2*\COR, -0.3*\COR) -- (0*\COR, -0.2*\COR);
	\draw[thick] (0*\COR, -0.2*\COR) -- (0*\COR, 0*\COR);
	

	
	%túlhevítő TH betűje
	\node[thick] at (1*\COR, -1*\COR) {TH};
	%gép
	\draw[thick] (5*\COR, -1.75*\COR) -- (6.5*\COR, -1.75*\COR);
	\draw[thick] (7*\COR, -1.75*\COR) circle (0.5cm*\COR);
	\draw[thick] (7*\COR, -1.25*\COR) -- (7*\COR, 0*\COR);
	\draw[thick] (6.66*\COR, -1*\COR) -- (7.33*\COR, -0.33*\COR);
	\draw[thick] (6.66*\COR, -0.85*\COR) -- (7.33*\COR, -0.19*\COR);
	\draw[thick] (6.66*\COR, -0.7*\COR) -- (7.33*\COR, -0.04*\COR);
	%gép betűje
	\node at (7*\COR, -1.75*\COR) {G};
	%pontok megrajzolása
	%1-es pont
	\draw[thick, fill = black] (0*\COR, -6*\COR) circle (\pontvastagsag*\COR);
	\node[circle, draw, inner sep = 1mm, thick] at (-0.5*\COR, -6*\COR) {1};
	%2-es pont
	\node[circle, draw, inner sep = 1mm, thick] at (0*\COR, -4*\COR) {2};
	%3-as pont
	\draw[thick, fill = black] (0*\COR, -1.9*\COR) circle (\pontvastagsag*\COR);
	\node[circle, draw, inner sep = 1mm, thick] at (1*\COR, -1.9*\COR) {3};
	%4-es pont
	\draw[thick, fill = black] (4*\COR, 0*\COR) circle (\pontvastagsag*\COR);
	\node[circle, draw, inner sep = 1mm, thick] at (4.5*\COR, 0*\COR) {4};
	%5-es pont
	\draw[thick, fill = black] (5*\COR, -3.2*\COR) circle (\pontvastagsag*\COR);
	\node[circle, draw, inner sep = 1mm, thick] at (5.5*\COR, -3.2*\COR) {5};
	%nyomaték megrajzolása
	\draw [->, thick] (5.5*\COR, -2.1*\COR) arc [radius = 0.45*\COR, start angle = -130, end angle = 130];
	%q_EL bejelölése
	\draw[->] (5.7*\COR, -4.5*\COR) -- (7*\COR, -4.5*\COR) node [midway, below] {$q_{EL}$};
	%q_BE bejelölése
	\draw[->] (-2*\COR, -4*\COR) -- (-0.7*\COR, -4*\COR) node [midway, above] {$q_{BE}$};
	%munka jelölése
	\node at (5.9*\COR, -1*\COR) {$w$};
	
	\end{tikzpicture}
	\caption{A túlhevítést alkalmazó körfolyamat kapcsolási vázlata}

\end{figure}
\pagebreak

\begin{equation*}
	q_{1-4} = h_4 - h_1,
	\quad
	q_{K} = h_5 - h_1,
	\quad
	w_t = h_4 - h_5,
	\quad
	\eta_T = \dfrac{w_t} {q_{BE}} = \dfrac{h_4-h_5} {h_4-h_1}
\end{equation*}
A $1-3$ szakasz izobár hőközlés\\
A $3-4$ szakasz izobár hőközlés (túlhevítés)\\
A $4-5$ adiabatikus expanzió\\
Az $5-1$ szakasz izoterm hőelvonás\\

A termodinamika első főtétele az energiamegmaradás törvénye.\\
Ez zárt, nyugvó rendszerre: $\Delta$$U=Q+W$ \\
azaz a rendszer belső energiájának változását hőközléssel/elvonással, illetve fizikai munkával tudjuk változtatni.\\
Ez differenciál alakban is felírható a törvény\\
\begin{equation*}
dU=\delta Q+ \delta W
\end{equation*} \\
ahol a belső energiát (állapotjelző) deriválhatjuk, míg az útfüggő mennyiségeknél csak a véges differenciákat vesszük.\\
Entalpia: térfogati munka + belső energia, azaz\\
\begin{equation*}
H=U+pV
\end{equation*}
Ez kis változásokra
\begin{equation*}
dH=dU+pdV+Vdp=\delta Q+\delta W_{fiz} +pdV+Vdp=\delta Q-pdV+Vdp=\delta Q+Vdp=\delta Q+\delta W_{tech}
\end{equation*}\\
$q_{1-4}$ =  $q_{BE}$\\
$dH = \delta Q$ \\
Az $1-4$ szakasz izobár, és állandó nyomás esetén az első főtétel adott alakja érvényes. Állandó nyomáson ($dp=0$ esetben) a bevitt hő megegyezik az entalpia változással.\\
$q_K$ = elvont hő\\
Az $5-1$ szakaszon végbemenő izoterm hőelvonásból következik, hogy A $T$ hőmérséklet állandó, vagyis $pV$ is állandó.A gáz belső energiája változatlan, így a gázzal közölt hő teljes egészében a térfogati munkára
fordítódik: 
\begin{equation*}
Q=W=\int\displaylimits_{V_1}^{V_2}p\dif V=\int\displaylimits_{V_1}^{V_2}\dfrac {p_1V_1}{V}\dif V=p_1V_1\ln{\dfrac{V_2}{V_1}}=p_1V_1\ln{\dfrac{p_1}{p_2}}
\end{equation*}

$w_t$ =  Munka\\
Állandó nyomáson ez megegyezik a munkára felhasználható belső
energiával, illetve az állandó nyomáson bevitt hővel.\\
$\eta_T$ =  Termikus hatásfok\\
A hasznos munka ($w$) és a bevitt hőmennyiség ($q_{BE}$) hányadosa.\\
Állapotjelzők:\\
\begin{tabular}{ | p{4cm} | p{3cm} | p{3cm} | } 
	\hline
\centering	Név &\centering Jelölés & Mértékegység \\ 
	\hline

\centering	Belső energia &\centering $U$ & $1\si{\joule=1\N\meter}$ \\ 
	\hline
\centering	Entalpia &\centering $H$ & $1\si{\joule=1\N\meter}$ \\ 
	\hline
\centering	Moláris belső entrópia &\centering $s$ & $\si{\kilo\joule\per\kilogram\kelvin}$ \\ 
	\hline
\centering	Moláris belső energia &\centering $u$ & $\si{\kilo\joule\per\kilogram\kelvin}$ \\ 
	\hline
\centering	Moláris entalpia &\centering $h$ & $\si{\kilo\joule\per\kilogram\kelvin}$ \\ 
	\hline
\centering	Nyomás &\centering $p$ & $1\si{\Pa=1\dfrac{\N}{\m^2}}$ \\ 
	\hline
\centering	Hőmérséklet &\centering $T$ & $1\si{\celsius={274.15}\kelvin}$ \\ 
	\hline
\end{tabular}
\pagebreak