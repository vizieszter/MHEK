\section*{6/24.feladat: Kompresszoros hűtőgép működése}

% Hozzáadás a tartalomjegyzékhez azonos címmel
\addcontentsline{toc}{section}{6/24.feladat: Kompresszoros hűtőgép működése}

% Táblázat a szerző adataival
\begin{tabular}{ | p{2cm} | p{14cm} | } 
	\hline
	Szerző & Hevesi Tamás (J3TV3W) \\ 
	\hline
	Szak & Anyagmérnöki alapszak. \\ 
	\hline
	Félév & 2019/2020 II. (tavaszi) félév \\ 
	\hline
\end{tabular}
\vspace{0.5cm}

% A feladat szövege
\noindent Mutassa be a kompresszoros ($NH_3$) hűtőgép működését expanziós gépes és fojtószelepes esetben! Rajzolja fel a hűtőkörfolyamatot T-s diagramban és a hűtőgép kapcsolási vázlatát! Ha mindegyik nevezetes pontban ismertek az állapotjelzők, akkor hogyan számítható a hűtőtérből elvont hő ($q_H$), a kompresszor sűrítési munkája $w_K$, a körfolyamatból elvezetett/a kondenzátorban leadott hő ($q_K$), az expanzós gép álal szolgáltatott munka ($w_T$) vagy az elvont hő csökkenése ($Δq_H$), és a fajlagos hűtőteljesítmény ($\varepsilon$)?

% A feladat megoldása
\subsubsection{Kompresszoros hűtőgép expanziós géppel(turbinával)}

\begin{figure}[ht]
	\begin{subfigure}[b]{0.5\textwidth}
	\centering
	\begin{tikzpicture} %Kapcsolási rajz(a)
	\draw[step=1cm,ultra thick] (0,0) to (1.4,0) to (1.65,0.4) to (1.9,-0.4) to (2.15,0.4) to (2.4,-0.4) to (2.65,0.4) to (2.9,-0.4) to (3.15,0.4) to (3.4,-0.4) to (3.65,0) to (5,0);
	\draw[step=1cm,ultra thick,->] (5,0) to (5,2);
	\draw[step=1cm,ultra thick] (5,2) to (5,4) to (4,3.5) to (4,2.5) to (5,2);
	\node[anchor=north] at (4.5, 3.25) {$K$}; 
	\node[anchor=north] at (-0.5, 3.25) {$T$};
	\draw[step=1cm,ultra thick,->] (4,3.5) to (4,5.5);
	\draw[step=1cm,ultra thick] (-1,5.5) to (1.25,5.5) to (1.5,5.9) to (1.75,5.1) to (2,5.9) to (2.25,5.1) to (2.5,5.9) to (2.75,5.1) to (3,5.9) to (3.25,5.1) to (3.5,5.5) to (4,5.5);
	\draw[step=1cm,ultra thick,->] (-1,5.5) to (-1,3.5);
	\draw[step=1cm,ultra thick] (-1,3.5) to (-1,2.5) to (0,2) to (0,0);
	\draw[step=1cm,ultra thick] (0,2) to (0,4) to (-1,3.5);
	\draw[step=1cm,ultra thick,->] (0,3) to (4,3);
	\node[anchor=north] at (1.5, 3.5) {$w_{T}$};
	\draw[ultra thick] (6.5, 3) circle (0.5);
	\node[anchor=north] at (6.5, 3.25) {$M$};
	\draw[step=1cm,ultra thick,->] (6,3) to (5,3);
	\node[anchor=north] at (5.5, 3.5) {$w_{K}$};
	\draw[step=1cm,ultra thick] (6.5,5) to (6.5,3.5);
	\draw[step=1cm,very thick] (6,4.3) to (7,4.6);
	\draw[step=1c,ultra thick] (1, -1) rectangle (4.05, 1);
	\draw[step=1c,ultra thick] (1.25, -0.75) rectangle (3.8, 0.75);
	\draw[step=1cm,ultra thick,->] (2.25,-1.6) to (2.25,-0.15);
	\node[anchor=north] at (2.55, -1.2) {$q_{H}$};
	\draw[step=1cm,ultra thick,->] (2.25,6) to (2.25,7.2);
	\node[anchor=north] at (2.6, 6.65) {$q_{K}$};
	\node[xshift={0mm}, yshift={-4mm}] at (0, 0) {$\Large\textcircled{\normalsize 4}$};	
	\node[xshift={0mm}, yshift={-4mm}] at (5, 0) {$\Large\textcircled{\normalsize 1}$};
	\node[xshift={0mm}, yshift={3mm}] at (4, 5.5) {$\Large\textcircled{\normalsize 2}$};
	\node[xshift={0mm}, yshift={3mm}] at (-1, 5.5) {$\Large\textcircled{\normalsize 3}$};	
	\fill[pattern=north east lines] (1, -1) rectangle (4.05, -0.75);
	\fill[pattern=north east lines] (1, -1) rectangle (1.25, 0.75);
	\fill[pattern=north east lines] (1, 0.75) rectangle (4.05, 1);
	\fill[pattern=north east lines] (3.8, -0.75) rectangle (4.05, 0.75);
	\end{tikzpicture}
	\caption{Kapcsolási vázlat}
\end{subfigure}
\begin{subfigure}[b]{0.5\textwidth}
	\centering
	\begin{tikzpicture}


% T-s diagram (a)
\begin{axis}[
width=8cm, height=8cm,
xmin=0, xmax=9,
ymin=195, ymax=475, 
axis lines = middle,
axis line style={->},
xlabel=$s \left(\si{\kilo\joule\per\kilogram\kelvin}\right)$, 
xlabel style={
	at=(current axis.right of origin), 
	anchor=north east
}, 
ylabel=$T \left(\si{\kelvin}\right)$, 
ylabel style={
	at=(current axis.above origin), 
	anchor=north east
},
xtick=\empty,
ytick=\empty,
]

	\addplot[thick] table {./j3tv3w/nh3fh.txt};
	\addplot[ultra thick, dashed, gray] table {./j3tv3w/nh3ib1.txt};
	\node[anchor=north east] at (axis cs:6.608, 450) {$p_K$};
	\addplot[ultra thick, dashed, gray] table {./j3tv3w/nh3ib2.txt}; 
	\node[anchor=north east] at (axis cs:7.920, 450) {$p_H$};
		\addplot[ultra thick, color=black,->=gray] coordinates {(5.64,239.562)(5.64,311.848)};
		\addplot[ultra thick, color=black,->=gray] coordinates {(5.64,311.848)(2.096,311.848)};
		\addplot[ultra thick, color=black,->=gray] coordinates {(2.096,311.848)(2.096,239.563)};
		\addplot[ultra thick, color=black,->=gray] coordinates {(2.096,239.563)(5.64,239.563)};
\node[anchor=north west] at (axis cs: 5.64, 239.562) {\pgfcircled{$1$}};
\filldraw[black, fill=black] (axis cs: 5.64, 239.562) circle (1mm);
\node[anchor=south west] at (axis cs: 5.64, 311.848) {\pgfcircled{$2$}};
\filldraw[black, fill=black] (axis cs: 5.64, 311.848) circle (1mm);
\node[anchor=south east] at (axis cs: 2.096, 311.848) {\pgfcircled{$3$}};
\filldraw[black, fill=black] (axis cs: 2.096, 311.848) circle (1mm);
\node[anchor=north east] at (axis cs: 2.096, 239.562) {\pgfcircled{$4$}};
\filldraw[black, fill=black] (axis cs: 2.096, 239.562) circle (1mm);

\addplot+[mark=none, domain=2.096:5.64, samples=100,
pattern=north east lines, draw = gray,
pattern color=black] coordinates {(2.096,239.652) (5.64,239.562)} \closedcycle;
\node[fill=white, rounded corners] at (axis cs: 3.8, 215) {$q_{H}$};

\end{axis}
	\end{tikzpicture}
	\caption{T-s diagram}
\end{subfigure}%
\caption{Kompresszoros hűtőgép expanziós géppel}

\end{figure}
\pagebreak

A hűtőgépet hidegfejlesztés ($q_H$) céljából üzemeltetjük, ami felírható az elpárologtatóból távozó és oda belépő hűtőközeg entalpiakülönbségéből. A kompresszorban munkabefektetéssel sűrítjük össze a hűtőközeget.Ez az egyetlen munkabefektetés az egész ciklus során.
\begin{itemize}
	\item a hűtőből elvont hő:\begin{equation*}
	q_H=h_1-h_4
	\end{equation*}
	\item a kompresszor által felhasznált munka:\begin{equation*}
	w_K=h_2-h_1
	\end{equation*}
	\item a körfolyamatokból elvezetett/ a kondenzátorban leadott hő:\begin{equation*}
	q_K=h_2-h_3
	\end{equation*}
\end{itemize}
A turbina nagy előnye, hogy működés közben a befektetett munka egy része visszanyerhető, így maximalizálható a fajlagos hűtőteljesítmény is.
\begin{itemize}
		\item a turbina által szolgáltatott munka:\begin{equation*}
		w_T=h_3-h_4
	\end{equation*}
	\item a fajlagos hűtőteljesítmény:\begin{equation*}
		\varepsilon= \dfrac{q_{H}}{w}=\dfrac{q_H}{w_K-w_T}=\dfrac{h_1-h_4}{h_2-h_1-(h_3-h_4)}
	\end{equation*}
\end{itemize} 


\subsubsection{Kompresszoros hűtőgép fojtószeleppel}

\begin{figure}[ht]
	\begin{subfigure}[b]{0.5\textwidth}
		\centering
		\usetikzlibrary{patterns.meta}
		\begin{tikzpicture}
		\draw[step=1cm,ultra thick] (-0.5,0) to (1.4,0) to (1.65,0.4) to (1.9,-0.4) to (2.15,0.4) to (2.4,-0.4) to (2.65,0.4) to (2.9,-0.4) to (3.15,0.4) to (3.4,-0.4) to (3.65,0) to (5,0);
		\draw[step=1cm,ultra thick,->] (5,0) to (5,2);
		\draw[step=1cm,ultra thick] (5,2) to (5,4) to (4,3.5) to (4,2.5) to (5,2);
		\node[anchor=north] at (4.5, 3.25) {$K$}; 
		\draw[step=1cm,ultra thick,->] (4,3.5) to (4,5.5);
		\draw[step=1cm,ultra thick] (-0.5,5.5) to (1.25,5.5) to (1.5,5.9) to (1.75,5.1) to (2,5.9) to (2.25,5.1) to (2.5,5.9) to (2.75,5.1) to (3,5.9) to (3.25,5.1) to (3.5,5.5) to (4,5.5);
		\draw[step=1cm,ultra thick,->] (-0.5,5.5) to (-0.5,3.5);
		\draw[step=1cm,ultra thick] (-1, 2) rectangle (0, 3.5);
		\draw[step=1cm,ultra thick,->] (-0.5,2) to (-0.5,0);
		\draw[ultra thick] (6.5, 3) circle (0.5);
			\draw[step=1cm,ultra thick,] (-0.5,3.5) to (-1,2);
			\draw[step=1cm,ultra thick,] (-0.5,3.5) to (0,2);
		\node[anchor=north] at (6.5, 3.25) {$M$};
		\draw[step=1cm,ultra thick,->] (6,3) to (5,3);
		\node[anchor=north] at (5.5, 3.5) {$w_{k}$};
		\draw[step=1cm,ultra thick] (6.5,5) to (6.5,3.5);
		\draw[step=1cm,very thick] (6,4.3) to (7,4.6);
		\draw[step=1c,ultra thick] (1, -1) rectangle (4.05, 1);
		\draw[step=1c,ultra thick] (1.25, -0.75) rectangle (3.8, 0.75);
		\draw[step=1cm,ultra thick,->] (2.25,-1.6) to (2.25,-0.15);
		\node[anchor=north] at (2.55, -1.2) {$q_{H}$};
		\draw[step=1cm,ultra thick,->] (2.25,6) to (2.25,7.2);
		\node[anchor=north] at (2.6, 6.65) {$q_{K}$};
		\node[xshift={0mm}, yshift={-4mm}] at (-0.5, 0) {$\Large\textcircled{\normalsize 4}$};	
		\node[xshift={0mm}, yshift={-4mm}] at (5, 0) {$\Large\textcircled{\normalsize 1}$};
		\node[xshift={0mm}, yshift={3mm}] at (4, 5.5) {$\Large\textcircled{\normalsize 2}$};
		\node[xshift={0mm}, yshift={3mm}] at (-0.5, 5.5) {$\Large\textcircled{\normalsize 3}$};	
		\fill[pattern=north east lines] (1, -1) rectangle (4.05, -0.75);
		\fill[pattern=north east lines] (1, -1) rectangle (1.25, 0.75);
		\fill[pattern=north east lines] (1, 0.75) rectangle (4.05, 1);
		\fill[pattern=north east lines] (3.8, -0.75) rectangle (4.05, 0.75);
		\end{tikzpicture}
		\caption{Kapcsolási vázlat}
	\end{subfigure}
	\begin{subfigure}[b]{0.5\textwidth}
		\centering
		\begin{tikzpicture}

		% a T-s diagram	(b)
		\begin{axis}[
	width=8cm, height=8cm,
	xmin=0, xmax=9,
	ymin=195, ymax=475, 
	axis lines = middle,
	axis line style={->},
	xlabel=$s \left(\si{\kilo\joule\per\kilogram\kelvin}\right)$, 
	xlabel style={
		at=(current axis.right of origin), 
		anchor=north east
	}, 
	ylabel=$T \left(\si{\kelvin}\right)$, 
	ylabel style={
		at=(current axis.above origin), 
		anchor=north east
	},
	xtick=\empty,
	ytick=\empty,
	]
	
	\addplot[thick] table {./j3tv3w/nh3fh.txt};
	\addplot[ultra thick, dashed, gray] table {./j3tv3w/nh3ib1.txt};
	\node[anchor=north east] at (axis cs:6.608, 450) {$p_K$};
	\addplot[ultra thick, dashed, gray] table {./j3tv3w/nh3ib2.txt}; 
	\node[anchor=north east] at (axis cs:7.920, 450) {$p_H$};
\addplot[ultra thick,color=black,->] table {./j3tv3w/tsh.txt};
	\addplot[ultra thick, color=black,->] coordinates {(5.64,239.562)(5.64,311.848)};
	\addplot[ultra thick, color=black,->] coordinates {(5.64,311.848)(2.096,311.848)};
	\addplot[ultra thick, color=black,->=] coordinates {(2.696,239.563)(5.64,239.563)};
	\node[anchor=north west] at (axis cs: 5.64, 239.562) {\pgfcircled{$1$}};
	\filldraw[black, fill=black] (axis cs: 5.64, 239.562) circle (1mm);
	\node[anchor=south west] at (axis cs: 5.64, 311.848) {\pgfcircled{$2$}};
	\filldraw[black, fill=black] (axis cs: 5.64, 311.848) circle (1mm);
	\node[anchor=south east] at (axis cs: 2.096, 311.848) {\pgfcircled{$3$}};
	\filldraw[black, fill=black] (axis cs: 2.096, 311.848) circle (1mm);
	\node[anchor=north east] at (axis cs: 2.696, 239.562) {\pgfcircled{$4^*$}};
	\filldraw[black, fill=black] (axis cs: 2.696, 239.562) circle (1mm);
	
	\addplot+[mark=none, domain=2.692:5.64, samples=100,
	pattern=north east lines, draw = gray,
	pattern color=black] coordinates {(2.692,239.652) (5.64,239.562)} \closedcycle;
	\node[fill=white, rounded corners] at (axis cs: 4.1, 215) {$q_{H}$};

		\end{axis}
		\end{tikzpicture}
		\caption{T-s diagram}
	\end{subfigure}
	\caption{Kompresszoros hűtőgép fojtószeleppel}
	
\end{figure}

A fojtás izentalpiás folyamat, tehát a kondenzátorból kifolyó hűtőközeg entalpiája fojtás előtt ($h_3$) és után ($h_4^*$) ugyanaz. A T-s diagramból rögtön látszik, hogy a hűtés ($q_H$) most kisebb, hiszen elesünk a visszanyert munkarésztől ($w_T$).

\begin{itemize}
	\item a hűtőből elvont hő csökkenése:\begin{equation*}
		\Delta q_H=h_4^*-h_4=h_3-h_4=w_T
	\end{equation*}
	\item a hűtőből elvont hő:\begin{equation*}
		q_H=h_1-h_4^*=h_1-h_3
	\end{equation*}
\end{itemize}
A fajlagos hűtőteljesítmény($\varepsilon$)  mindig a nyert hideg ($q_H$) és a munka ($w$) viszonya. Ebben az esetben a munka ($w$) megegyezik a kompresszor által felhasznált munkával ($w_K$).
\begin{equation*}
	\varepsilon= \dfrac{q_{H}}{w}=\dfrac{q_h}{w_K}=\dfrac{h_1-h_3}{h_2-h_1}
\end{equation*}
A fojtószeleppel szerelt kompresszoros hűtőgép esetében kisebb hőt tudunk elvonni a környezetből, de a kondenzátorban leadott hő és a kompresszor által felhasznált munka változatlan marad.


\pagebreak