
% A feladat címe automatikus számozás nélkül
\section*{4.21.feladat: Túlhevítést alkalmazó Rankine-Clausius-körfolyamat,a tápszivattyú hatásának figyelembe vételével}

% Hozzáadás a tartalomjegyzékhez azonos címmel
\addcontentsline{toc}{section}{4.21. feladat: Túlhevítést alkalmazó Rankine-Clausius-körfolyamat, a tápszivattyú hatásának figyelembe vételével}

% Táblázat a szerző adataival
\begin{tabular}{ | p{5cm} | p{14cm} | } 
	\hline
	Grőber Adél HRCJNO\\ 
	\hline
	Vegyészmérnök alapszak\\ 
	\hline
	2019/2020 II. (tavaszi) félév \\ 
	\hline
\end{tabular}
\vspace{0.5cm}

% A feladat szövege
\noindent Rajzolja le a túlhevítést alkalmazó Rankine–Clausius-körfolyamat kapcsolási vázlatát, a körfolyamatot T-s diagramban, figyelembe véve a tápszivattyú hatását! Jelölje be a munkát ($w$) és a kondenzátorban elvont hőt ($q_K$)! Ha mindegyik nevezetes pontban ismertek az állapotjelzők, akkor hogyan számítható a bevitt hő ($q_{BE}$), a munka ($w$), a kondenzátorban elvont hő ($q_K$) és a termikus hatásfok ($\eta_T$)?


% A feladat megoldása
\begin{center}

\begin{itemize}
	\item 	hőerőművek vízgőz körfolyamatát írja le a Rankine-Clausius-körfolyamat
	\item a körfolyamat elvégzéséhet szükséges 4 folyamat:
	\begin{itemize}
		\item állandó nyomású hőközlés a kazánban
		\item adiabatikus kiterjedés/expanzió a hőerőgépben (turbina)
		\item állandó nyomáson történő hőelvonás, illetve lecsapódás/kondenzáció a kondenzátorban
		\item a víz adiabatikus visszaszivattyúzása a kazánba
		\
	\end{itemize}
\end{itemize}
\vspace{1 cm}	
$w$: hasznos munka  $q_{EL}$:  veszteség 	$w_{t}$: turbina munkája 		$w_{sz}$: szivattyú munkája
\end{center}

\begin{equation}
	w=q_{BE}-q_{EL}
\end{equation}
\begin{equation}
	q_{BE}=w+q_{EL}
\end{equation}		
\begin{equation}
	q_{BE}=q_{1-4}=q_{4}-h_1 (p= \textrm{áll}.)
\end{equation}
\begin{equation}
	q_{EL}=-q_{5-6}=h_5-h_6
\end{equation}
\begin{equation}
w_{t}=h_4-h_5  (\delta \ wq_{t}=-dh)
\end{equation}
\begin{equation}
w_{sz}=h_1-h_6\approx 0,01…0,03 w_{t}
\end{equation}
\begin{equation}
\eta_T{} = \dfrac{w_{t}-w_{sz}}{q_{BE}} \approx 35\%…45\%
\end{equation}

\begin{figure}[!ht]
\centering
\label{figure:sm}
\begin{tikzpicture}
%alap ábra
\draw[->](0,0) -- (4,0) ;
\draw (4,0) -- (8,0) -- (8,-4) -- (10,-5) -- (10, -7.5);
%turbina
	\draw(8,-3) -- (10, -2) -- (10, -5);
	\draw(10, -3.5) -- (12, -3.5);
	\draw(13, -3.5) circle(1);
	\draw(13, -2.5) -- (13, 1);
	\draw(12.5, -2) -- (13.5, -1.5);
	\draw(12.5, -1.25) -- (13.5, -0.75);
	\draw(12.5, -0.5) -- (13.5, 0);
	\begin{scope}[yscale=-1,xscale=1]
			\draw[->] (10.5, 3.25) {[rotate=-240] arc [radius=0.5, start angle=80, end angle=380 ]};
	\end{scope}
%alja
\draw(10,-9.5) -- (10,-10);
\draw [->] (10,-10) -- (5, -10);
\draw (5, -10) -- (0, -10) -- (0, -9.5);
%összekötés	
	\draw (0, -2.1) -- (0,0);
%tápszivattyú	
	\draw(0, -8.5) circle(1);
%összekötés		
	\draw(0, -7.5) -- (0, -7);
%kazán	
	\draw(-1, -7) -- (1, -7) -- (1, -4) -- (-1, -4) -- (-1, -7);
%túlhevítő vízgőz	
	\draw (0, -3.4) -- (-0.5, -3.25) -- (0.6, -3) -- (-0.6, -2.75) -- (0.6, -2.5) -- (-0.5, -2.25) -- (0, -2.1);
%összekötés		
	\draw (0, -4) -- (0, -3.4);
%kondenzátor
\draw(10, -8.5) circle(1);
\draw (12, -9) -- (9.6, -8.75) -- (10.2, -8.5) -- (9.6, -8.25) -- (12, -8);

\node[anchor=west] at (10, -10) {$S$};
\node[anchor = south] at (9.125, -3.75) {$T$};
\node[anchor = mid] at (13, -3.5) {$G$};
\node[anchor = east] at (-1, -2.5) {$TH$};
\node[anchor = east] at (0.3, -5.5) {$K$};
\node[anchor = east] at (-1.5, -8.5) {$SZ$};

\node[anchor = mid] at (0, -8.5) {$\bigtriangleup$};



	
\end{tikzpicture}
\caption{Kapcsolási rajz}
\end{figure}

%ÁBRA 2
\begin{figure}[h]
	\centering
	\label{figure:vgtsd}
	\begin{tikzpicture}
	
	% Rács és vágómaszk
	\draw[step=1cm, gray, very thin] (-1.5, -1) grid (14.5, 11);;
	\clip (-1.5, -1) rectangle (14.5, 11);
	
	% A tengelykeresztet az axis környezet hozza létre
	\begin{axis}[
			width=16cm, height=12cm,
			xmin=0, xmax=10.8,
			ymin=0, ymax=475, 
			axis lines = middle,
	axis line style={->},
	xlabel=$s \left(\si{\kilo\joule\per\kilogram\kelvin}\right)$, 
	xlabel style={
		at=(current axis.right of origin), 
		anchor=north east
	}, 
	ylabel=$T \left(\si{\degreeCelsius}\right)$, 
	ylabel style={
		at=(current axis.above origin), 
		anchor=north east
	},
	xtick={1, 2, 3, 4, 5, 6, 7, 8, 9},
	ytick={100, 200, 300, 400}
	]
	
	\addplot[thick] table {./tshf.txt};
	
	\end{axis}

	% Folyamat
	\draw[->, ultra thick] (0.5, 2) -- (3.44, 5);
	\node[anchor=north] at (3.24, 5.5) {$2.$};
	\draw[->, ultra thick] (3.44, 5) -- (8.34, 5);
	\node[anchor=north] at (8.34, 4.7) {$3.$};
	\draw[->, ultra thick] (8.34, 5) -- (10, 9);
	\node[anchor=north] at (10.4, 9) {$4.$};
	\draw[->, ultra thick] (10, 9) -- (10, 1);
	\node[anchor=north] at (10.3, 0.9) {$5.$};
	\draw[->, ultra thick] (10, 1.1) -- (0.9, 1.1);
	\node[anchor=north] at (0.9, 0.8) {$6.$};
	\draw[->, ultra thick] (0.9, 1.1) -- (0.5, 2);
	\node[anchor=north] at (0.4, 2.5) {$1.$};
	
	\end{tikzpicture}
	\caption{Rankine-Clausius-körfolyamat T-s diagramja, túlhevítést alkalmazva, a tápszivattyú hatásának figyelembe vételével}
\end{figure}

% Oldaltörés
\pagebreak